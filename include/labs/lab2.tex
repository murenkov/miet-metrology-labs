\chapter{Лабораторная работа №2 (4/2) \\
Панорамный измеритель КСВН}

\section{Цель работы}

Изучить теоретические основы измерений коэффициентов отражения и ослаблений на СВЧ:
\begin{itemize}
    \item определения измеряемых величин;
    \item единицы величин;
    \item принцип и метод измерений величин <<по определению>>;
    \item операции, выполняемые при измерении;
    \item физические явления, выражающие погрешности измерений.
\end{itemize}

% \section{Теоретические сведения}

\section{Практическая часть}

Согласно методике, проведём измерения и занесём результат в табл.~\ref{tab:reflection-coefficients}.

\begin{table}[H]
    \centering
    \caption{--}%
    \label{tab:reflection-coefficients}
    \resizebox{\textwidth}{!}{%
        \begin{tabular}{@{}lccccccc@{}}
            \toprule
            Частота установленная, МГц    & 4.6 & 4.7 & 4.8 & 4.9 & 5.0  & 5.1  & 5.2  \\ \midrule
            $P_\text{ХХ}$, дБм            & 0.5 & 0.5 & 0.5 & 0.2 & 0.2  & 0.8  & 1.2  \\
            $P_\text{КЗ}$, дБм            & 1.2 & 1.2 & 1.1 & 0.5 & -0.1 & -0.4 & -0.2 \\
            $P_\text{-6~дБ}$, дБм         & 1.0 & 0.9 & 0.7 & 0.2 & 0.1  & 0.5  & 0.5  \\
            $P_\text{-6~дБ + КЗ}$, дБм    & 0.7 & 0.5 & 0.2 & 0.7 & 0.5  & 0.2  & 0.1  \\
            $P_\text{-10~дБ}$, дБм        & 0.8 & 0.6 & 0.3 & 0.2 & 0.2  & 0.4  & 0.5  \\
            $P_\text{-10~дБ + КЗ}$, дБм   & 0.6 & 0.6 & 0.6 & 0.5 & 0.3  & 0.3  & 0.4  \\
            $P_\text{Я2М-64 150~Ом}$, дБм & 0.9 & 0.9 & 0.6 & 0.2 & 0.4  & 0.5  & 0.2  \\
            $P_\text{Я2М-64 240~Ом}$, дБм & 0.9 & 0.9 & 0.6 & 0.2 & 0.4  & 0.5  & 0.2  \\
            $P_\text{Я2М-64 400~Ом}$, дБм & 0.9 & 0.9 & 0.6 & 0.2 & 0.4  & 0.5  & 0.2  \\ \bottomrule
        \end{tabular}
    }
\end{table}

\section{Ответы на вопросы}

\begin{enumerate}
    % 5, 19, 20
    \item
        Перечислить линейные преобразования напряжения СВЧ в ПИКО.

        \emph{Ответ.}

    \item
        Дать определения ослабления как измеряемой величины в разах, в децибелах.
        Какие составные части в ПИКО являются измерителями проходящей мощности, падающей мощности?

        \emph{Ответ.}

    \item
        Зависимости переходного ослабления С[дБ] направленных ответвителей падающей и отраженной волн (в ПИКО в полосе частот на рисунках а/б).
        Чему равна максимальная относительная погрешность измерения КСВН в \%, ослабления в дБ.
        \begin{figure}[H]
        \end{figure}

        \emph{Ответ.}

\end{enumerate}
