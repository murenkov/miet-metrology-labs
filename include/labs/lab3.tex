\chapter{Лабораторная работа №3 (2Ag)\\
Измерение метрологических характеристик генератора}

\section{Цель работы}

\begin{enumerate}
    \item Ознакомиться с измерительными приборами, такими как анализатор спектра, генератор сигналов и ваттметр, входящими в состав лабораторной установки, и программами удалённого управления ими.
    \item Изучить методы определения метрологических характеристик генератора сигналов.
    \item Составить протоколы, содержащие результаты измерения частоты генератора и мощности на его выходе.
\end{enumerate}

% \section{Теоретические сведения}

\section{Практическая часть}

В соответствии с методикой выполним измерения частот генератора и, найдя значение относительной погрешности, занесём результаты в табл.~\ref{tab:frequency-relative-error}.

\begin{table}[H]
    \centering
    \caption{Определение абсолютной погрешности установки мощности}%
    \label{tab:frequency-relative-error}
    \resizebox{\textwidth}{!}{%
        \begin{tabular}{@{}ccc@{}}
            \toprule
            \multicolumn{1}{l}{Частота установленная, МГц} & Частота измеренная, Гц & Относительная погрешность частоты \\ \midrule
            \rowcolor[HTML]{EFEFEF}
            1                                              & 1000000.090            & $0.9 \cdot 10^{-7}$               \\
            100                                            & 100000009.720          & $0.97 \cdot 10^{-7}$              \\
            \rowcolor[HTML]{EFEFEF}
            1000                                           & 1000000097             & $0.97 \cdot 10^{-7}$              \\
            2000                                           & 2000000196             & $0.98 \cdot 10^{-7}$              \\
            \rowcolor[HTML]{EFEFEF}
            3000                                           & 3000000290             & $0.97 \cdot 10^{-7}$              \\ \bottomrule
        \end{tabular}%
    }
\end{table}

В соответствии с методикой выполним измерения частот генератора и, найдя значение относительной погрешности, занесём результаты в табл.~\ref{tab:power-absolute-error}.

\begin{table}[H]
    \centering
    \caption{Определение абсолютной погрешности установки мощности}%
    \label{tab:power-absolute-error}
    \resizebox{\textwidth}{!}{%
        \begin{tabular}{@{}cccc@{}}
            \toprule
            Частота установленная, МГц & Мощность установленная, дБм & Мощность измеренная, дБм & Абсолютная погрешность установки мощности, дБм \\ \midrule
            \multirow{5}{*}{100}       & 0                           & 0.07                     & 0.07                                           \\
                                       & -10                         & -9.95                    & 0.05                                           \\
                                       & -20                         & -19.93                   & 0.07                                           \\
                                       & -30                         & -29.94                   & 0.06                                           \\
                                       & -40                         & -39.94                   & 0.06                                           \\
            \multirow{5}{*}{1000}      & 0                           & 0.09                     & 0.09                                           \\
                                       & -10                         & -9.94                    & 0.06                                           \\
                                       & -20                         & -19.91                   & 0.09                                           \\
                                       & -30                         & -29.91                   & 0.09                                           \\
                                       & -40                         & -39.92                   & 0.08                                           \\
            \multirow{5}{*}{2000}      & 0                           & 0.08                     & 0.08                                           \\
                                       & -10                         & -9.92                    & 0.08                                           \\
                                       & -20                         & -19.92                   & 0.08                                           \\
                                       & -30                         & -29.92                   & 0.08                                           \\
                                       & -40                         & -39.93                   & 0.07                                           \\ \bottomrule
        \end{tabular}%
    }
\end{table}

\section{Ответы на вопросы}

\begin{enumerate}
    \item
        Что такое генератор сигналов и для чего он нужен?

        \emph{Ответ.}
        Генератор сигналов --- это устройство, позволяющее получать сигнал определенной природы (электрический, акустический и т.д.), имеющий заданные характеристики (форму, энергетические или статистические характеристики и т.д.).
        Генератор сигналов нужен для генерации сигналов.

    \item
        Что такое сигнал?

        \emph{Ответ.}
        Сигнал --- изменяющаяся физическая величина, отображающая сообщение или предназначенная для функционирования технического средства.

    \item
        Что такое метрологические характеристики?

        \emph{Ответ.}
        Метрологические характеристики --- это характеристики свойств средства измерений, оказывающие влияние на результат измерения и его погрешности.

    \item
        Какова цель измерений?

        \emph{Ответ.}
        Цель измерений --- определение истинных значений постоянной или изменяющейся измеряемой величины.

    \item
        Как вычислить погрешность измерения частоты частотомером и мощности ваттметром?

        \emph{Ответ.}
        Для вычисления абсолютной погрешности находят разницу между измеренным значением и действительным значением.
        Для вычисления относительной погрешности находят отношение абсолютной погрешности к действительному значению.
\end{enumerate}
